 
% !TeX spellcheck = pl_PL
\documentclass[]{article}
\usepackage[polish]{babel}
\usepackage[utf8]{inputenc}
\usepackage[T1]{fontenc}
\usepackage{array}
\usepackage{amsmath}
\usepackage{graphicx}
\usepackage{multirow}
\usepackage{makecell}
\usepackage{geometry}
\graphicspath{{./images/}}


\newgeometry{vmargin={30mm}, hmargin={30mm}}
	\begin{document}
		\begin{table}[h]
		\centering
		\begin{tabular}{|c|c|c|c|c|c|}
			\hline 
			\makecell{Nr. \\ ćwicz.}& \makecell{Data} & \makecell{Imię i nazwisko} & \makecell{Wydział} & \makecell{Semestr} & \makecell{Grupa I1\\nr. lab.} \\ 
			121 & 29 listopada 2019 & \makecell{Jakub Gosławski 141222\\Michał Wiśniewski 141355} & Informatyki & 3 & 5 \\ 
			\hline 
			\multicolumn{3}{|l|}{Prowadzący: Wojciech Marciniak} &  &  \multicolumn{2}{l|}{Ocena:}  \\
			\hline
		\end{tabular}
	\end{table} 

	Temat ćwiczenia: Badanie rezonansu mechanicznego
	
	\section{Podstawy Teoretyczne}
	\subsection{Wzory}
	\begin{align}
	\beta = \frac{1}{T}ln\frac{A_n}{A_{n+1}}\\
	\omega_0 = \frac{2\pi}{T}\\
	\omega '=\sqrt{\omega^2_0 = \beta^2}\\
	\tau = \frac{1}{2\beta}\\
	Q = \omega_0\tau = \frac{\omega_0}{2\beta}
	\end{align}
	(1) wspólczynnik wytłumienia
	(2,3) częstotliwość kołowa
	(4) czas relaksacji
	(5) dobroć oscylatora
	\section{Wyniki Pomiarów i  Obliczenia}
	\subsection{Elektromagnes $0V$}
	Zmierzony czas 10 wachnięć - $17.01s$\\
	Okres $T = \frac{17.01s}{10}=1.70s$\\
	$\omega = 3.69 \left[ \frac{rad}{s}\right]$\\
	$\omega' = 3.69 \left[ \frac{rad}{s}\right]$\\
	Ponieważ dla tej wartości napięcia w elektromagnesie wartość $\beta$ jest tak mała że po zaokrągleniu $\omega=\omega'$\\
	Zmierzone amplitudy kolejnych wachnięć i obliczone współczynniki tłumienia
	\begin{table}[h]
		\begin{tabular}{|c|c|}
			\hline 
			A[cm] & $\beta\left[ \frac{1}{s}\right] $ \\ 
			\hline 
			18.0 & 0.00657 \\ 
			\hline 
			17.8 & 0.00664 \\ 
			\hline 
			17.6 & 0.00672 \\ 
			\hline 
			17.4 & 0.00680 \\ 
			\hline 
			17.2 & 0.00668 \\ 
			\hline 
		\end{tabular} 
	\end{table}

	$\beta_{\text{śr}} = 0.00668 \left[ \frac{1}{s}\right] $
	
	$\tau=74.83[s]$
	
	$Q = 276.41$
	
	\subsection{Elektromagnes $10V$}
	Zmierzony czas 10 wachnięć - $17.49s$\\
	Okres $T=\frac{17.49s}{10} = 1.75s$\\
	$\omega = 3.59 \left[ \frac{rad}{s}\right]$\\
	$\omega' = 3.59 \left[ \frac{rad}{s}\right]$\\
	Zmierzone amplitudy kolejnych wachnięć i obliczone współczynniki tłumienia
	\begin{table}[h]
		\begin{tabular}{|c|c|}
			\hline 
			A[cm] & $\beta\left[ \frac{1}{s}\right] $ \\ 
			\hline 
			18.0 & 0.112 \\ 
			\hline 
			14.8 & 0.191 \\ 
			\hline 
			10.6 & 0.175 \\ 
			\hline 
			7.8 & 0.189 \\ 
			\hline 
			5.6 & 0.167 \\ 
			\hline 
		\end{tabular} 
	\end{table}

	$\beta_{\text{śr}} = 0.167 \left[ \frac{1}{s}\right] $
	
	$\tau=2.996[s]$
	
	$Q = 10.76$
	
	\subsection{Elektromagnes $10V$}
	Zmierzony czas 3 wachnięć - $5.33s$, po 3 wachnięciach wachadło zatrzymało się\\
	Okres $T=\frac{5.33s}{3} = 1.77s$\\
	$\omega = 3.53 \left[ \frac{rad}{s}\right]$\\
	$\omega' = 3.43 \left[ \frac{rad}{s}\right]$\\
	Zmierzone amplitudy kolejnych wachnięć i obliczone współczynniki tłumienia
	\begin{table}[h]
		\begin{tabular}{|c|c|}
			\hline 
			A[cm] & $\beta\left[ \frac{1}{s}\right] $ \\ 
			\hline 
			18.0 & 0.582 \\ 
			\hline 
			6.4 & 0.655 \\ 
			\hline 
			2.0 & 0.1.296 \\ 
			\hline 
			0.2 & 0.844 \\ 
			\hline 
		\end{tabular} 
	\end{table}
	
	$\beta_{\text{śr}} = 0.844 \left[ \frac{1}{s}\right] $
	
	$\tau=0.592[s]$
	
	$Q = 2.09$
	\section{Dyskusja Błędów Pomiarowych}
	\section{Wnioski}
	\section{Wykresy}
\end{document}