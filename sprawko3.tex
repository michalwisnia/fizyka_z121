 
% !TeX spellcheck = pl_PL
\documentclass[]{article}
\usepackage[polish]{babel}
\usepackage[utf8]{inputenc}
\usepackage[T1]{fontenc}
\usepackage{array}
\usepackage{amsmath}
\usepackage{graphicx}
\usepackage{multirow}
\usepackage{makecell}
\usepackage{geometry}
\graphicspath{{./images/}}


\newgeometry{vmargin={30mm}, hmargin={30mm}}
	\begin{document}
		\begin{table}[h]
		\centering
		\begin{tabular}{|c|c|c|c|c|c|}
			\hline 
			\makecell{Nr. \\ ćwicz.}& \makecell{Data} & \makecell{Imię i nazwisko} & \makecell{Wydział} & \makecell{Semestr} & \makecell{Grupa I1\\nr. lab.} \\ 
			121 & 29 listopada 2019 & \makecell{Jakub Gosławski 141222\\Michał Wiśniewski 141355} & Informatyki & 3 & 5 \\ 
			\hline 
			\multicolumn{3}{|l|}{Prowadzący: Wojciech Marciniak} &  &  \multicolumn{2}{l|}{Ocena:}  \\
			\hline
		\end{tabular}
	\end{table} 

	Temat ćwiczenia: Badanie rezonansu mechanicznego
	
	\section{Podstawy Teoretyczne}
	\section{Wyniki Pomiarów}
	\section{Obliczenia}
	\section{Dyskusja Błędów Pomiarowych}
	\section{Wnioski}
	\section{Wykresy}
\end{document}